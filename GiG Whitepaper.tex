\documentclass{article}
\usepackage{amsmath}
\usepackage{hyperref}
\title{GiG: A Decentralized Platform for the Gig Economy}
\author{Ishtar Eve}
\date{February 2019}
\begin{document}
   \maketitle
   \begin{abstract}
A decentralized application for the labor marketplace that connects employers with independent contractors is described. GiG is intended to reduce friction and to eliminate fees collected by employment agencies, recruiting platforms, and financial institutions. Common bugs will be avoided by using Plutus, the strongly typed functional programming language of the Cardano blockchain. 

\paragraph{} The native currency of the system is the Gig Economy Token (GET). The algorithm used for its creation prevents speculative bubbles, and funds a Treasury System DAO (Decentralized Autonomous Organization). The DAO finances expenses for the ecosystem through proposals created and voted by the GiG community.

\paragraph{} GiG offers multiple tools that will give value to freelancers and employers, gaining the ability to interact without the need of trust, transacting on a peer to peer basis (without using financial institutions).

\paragraph{}(Specific information about GiG functionalities on this section using bulletpoints).

\paragraph{} Note: GiG is under construction. New versions of this paper will be updated periodically.

\end{abstract}

\section{Introduction}

(Current state of the gig economy marketplace, problems that the GiG protocol solves.)

A study by Intuit predicts that the gig economy will be about 43\% of the workforce by 2020.

(Paper organization)

(Thoughts)
\paragraph{} All around the world, there's lots of work that needs to be done, and many people that need to work. Unfortunately, the people who are willing to pay a certain amount of money for a certain work to be done, very often can't connect with the people who would benefit from the laboral opportunity and who would be willing to do the work in exchange for such amount.

\section{GET Token Creation}
GET tokens are created when ADA is received by the GET creation smart contract (GCSM).
The amount of GET created and awarded to the ADA sender's wallet $\alpha$, will be an arbitrary constant $\phi$ (currently set to 1000 GET), divided by the block height $\ell$, starting from the first block since the GET creation smart contract gets published. That is,
\[ \alpha
  = \dfrac{\phi}{\ell}
\]

\section{Treasury System and DAO funding}
All the ADA received by the GCSM is managed by a treasury system DAO based on the research made by IOHK for the Zendao \cite{zhangb2}.
 
\section{Creating Job Offers}
An employer signs a Job Offer Posting Transaction (JOPT) with his Cardano private key. The costs of these transactions are paid with GET. These transactions will always include an amount of GET offered for the task and a specific job description. A JOPT might also include other information about the task, such as a location list, a expected schedule, and other instructions or information about the task.

\section{Applying for Job Offers}
When a freelancer finds a job offer he is interested in, he will send an application transaction to the employer.

\section{Accepting Applications}
When an employer receives applications for a job offer he created, he can accept said applications, which would create an escrow for the amount detailed in his job offer.

\section{Escrow Release}
Once the requirements of the job have been fulfilled by the freelancer, the employer can release the GET from the escrow.

\section{Dispute Resolution}
A dispute can be started by either an employer or by an employee. The arbiter agreed upon before the escrow creation receives a notification of the start of the dispute, and needs to review any evidence provided by the employer and by the employee, and release the escrow in favor of the actor whom he deems is in the right.

Geolocation

DAO funding


Creating Job Offers



Escrow Creation

Booking inquiries
Booking a Freelancer

Making a transaction
Management Tools
Automatic Timesheets
Communication Between Parties
Job Completion


\begin{thebibliography}{99}

\bibitem{zhangb2}
\href{https://www.lancaster.ac.uk/staff/zhangb2/treasury.pdf}{A Treasury System for Cryptocurrencies:Enabling Better Collaborative Intelligence}
Bingsheng Zhang1, Roman Oliynykov2, and Hamed Balogun3

\end{thebibliography}

\end{document}
